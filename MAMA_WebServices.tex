% @author: MetaboHUB
% @date: 2016/03/11
% @note: this document is versioned through a GIT repository; sharing is caring

\documentclass[a4paper,11pt]{article}

%%%%%%%%%%%%%%%%%%%
% Polices 
\renewcommand*\sfdefault{cmss}
\renewcommand*\familydefault{\sfdefault}  %% Only if the base font of the document is to be sans serif

%%%%%%%%%%%%%%%%%%%
% packages
\usepackage{cmbright}
\usepackage{algorithm}
\usepackage{algorithmic}
\usepackage{amssymb}
\usepackage{amsmath}
\usepackage{pifont}
\usepackage[usenames,dvipsnames]{color}

\usepackage{wallpaper}
\usepackage[utf8]{inputenc} % accents
\usepackage[english]{babel}
\usepackage{fixltx2e}

\usepackage{array}
\usepackage{color}

% margins
\usepackage{geometry}
\geometry{left=2.3cm,right=2.3cm,top=3cm,bottom=2cm}

% pdf hyperlinks
\usepackage{hyperref}

% PATH FOR LOGO / IMAGES,
\graphicspath{ {mth_template/logos/} }

%%%%%%%%%%%%%%%%%%%%%%%%%%%%%%%%%%%%%%%%%%%%%%%%%%%%%%%%%%%%%%%%%%%%% metadata
\usepackage{hyperref}
\hypersetup{
	pdftitle={MAMA - WebServices Guide},
	pdfauthor={MetaboHUB - WP4},
	colorlinks=true,
	urlcolor=blue,
	linkcolor=black,
	citecolor=black,
}

% dates
\usepackage{datetime}
\usepackage{multirow}
\usepackage{tabularx}
\usepackage{hyperref}
\usepackage{pifont}

% italic text
\newcommand{\ie}{\textit{i.e.}~}
\newcommand{\eg}{\textit{e.g.}~}
%\newcommand{\exp}{\textit{exp}~}
\newcommand{\cf}{\textit{cf.~}}
\newcommand{\textitt}[1]{\textit{\texttt{#1}}}
\def\myversion{0.1}

%tickmarks
\newcommand{\tick}{\textcolor{ForestGreen}{\ding{52}}}
\newcommand{\tickNo}{\hspace{1pt}\textcolor{BrickRed}{\ding{55}}}

%colors
%\usepackage[table]{xcolor}
\definecolor{inraGreen}{rgb}{0.50196078431,0.62745098039,0}
\definecolor{mthRed}{rgb}{1,0.00784313725,0.00784313725}
%%%%%%%%%%%%%%%%%%%

% meta-data
\author{MetaboHUB - WP3}
\date{2014}

%%%%%%%%%%%%%%%%%%%
% title page 
\newcommand{\HRule}{\rule{\linewidth}{0.5mm}}
\usepackage{footnote}
\makesavenoteenv{tabular}
\makesavenoteenv{table}
\newcommand{\specialcell}[2][c]{%
  \begin{tabular}[#1]{@{}c@{}}#2\end{tabular}}

% version
\input{vc.tex}
%%% This file is edited automatic script.
%%% Do not edit this file!
%%%
	\gdef\ReleaseVersion{1.4.0}


\begin{document}

%%%%%%%%%%%%%%%%%%%%%%%%%%%%%%%%%%%%%%%%%%%%%%%%%%%%%%%%%%%%%%%%%%%%% title page
\input{./files/tex/title_doc_ws_mama.tex}

\input{mth_template/blocks/body.tex}

%%%%%%%%%%%%%%%%%%%%%%%%%%%%%%%%%%%%%%%%%%%%%%%%%%%%%%%%%%%%%%%%%%%%% START

%\section{Document History}
%
%\begin{itemize}
%	\item 2014/02/20: v 0.1~: document template (init; main sections).
%\end{itemize} 

\section{Introduction}
%\subsection{About}
\hspace*{\parindent}
As a part of the MetaboHUB project, 
the WebService access to the MAMA service is integrated in the project's forth Workpackage. 
It is possible for anyone to develop his own client component in order to call the WebService directly. 
The WebServire base URL is:  \href{https://mama-rest.metabohub.fr/}{mama-rest.metabohub.fr}. 
\newline

\subsection{Project scope and users authorizations}
\hspace*{\parindent}
The MAMA service manage both users and projects. 
Users may have different rights (user, project manager, admin);
some method are only available for users with the correct grants 
(\eg only administrators can promote a user with ``project manager'' or ``admin'' rights). 
A ``project manager'' has the same rights as a ``user'' and an ``admin'' also has the ``project manager'' grants. 
All method require an user token (expect of course ``POST /user'' and ``GET /token'' methods). %note: ``a user'' or ``an user''? grammar said ``an'' but it's sound better with ``a''
The token must be send as GET parameter to each methods who require an authentication 
(event for POST / PUT / DELETE requests) 
like ``\&token=YOUR\_TOKEN\_VALUE''. 
Tokens are destroy after 14 days. 
Only one token per user is allowed: 
if a new one is requested, the order one is destroyed. 
\newline

\subsection{REST standards and best practices}
\hspace*{\parindent}
As a standard REST webservice, a same web resource, identified by Uniform Resource Identifiers (URIs), 
can be call with different HTTP verbs (GET, POST, PUT or DELETE); 
\eg ``GET /user/\{id\}'' will show the user attributes, ``PUT /user/\{id\}'' will update the users informations and settings. 
To view or edit a project, you must be the project owner or a ``project manager''. 
MAMA check the security and authorization thank to the token management system.  
\newline

\hspace*{\parindent}
The webservice can return different three formats: JSON and XML objects or just flat text. 
You can set the webservice output format thanks ``Accept'' HTTP header or just add ``format=xxx'' GET parameter in your request 
(event for POST / PUT / DELETE requests).
\newline

\hspace*{\parindent}
Special note about LDAP login: if you use LDAP authentication you do not need to create an account with ``POST /user'' method. 
Use ``POST /token'' method with your LDAP login / passwords. 

%\hspace*{\parindent}
%The academic institutions are allowed to use data for no-commercial works;~
%private institutions may contact us first.
%The whole WebService and its source code are freely available, 
%thus allowing academic institutions to deploy it on their own network and adapt it to their specific needs.

%\newline

%All webservices requests examples are based on current MetaboHUB Spectral Database main portal.

%\subsection{Definitions}
%
%\begin{itemize}
%	\item WebService:.
%\end{itemize}

\section{Web Service REST Requests}
\hspace*{\parindent}
%All the following example are based on the main instance of the Spectral Database REST WebServices.
%If you deploy your own instance of the Spectral Database, adapt the URL of the request.

First you must create your account (see ``POST /user'' in section \ref{postUser} page \pageref{postUser}) then authenticate yourself (see ``POST /token'' in section \ref{postToken} page \pageref{postToken}).

\subsection{``Anonymous'' rights}

\subsubsection{get MAMA informations}
\begin{tabular}{ | l | l | }
	\hline
	HTTP verb and URI & \texttt{GET /} \\
	\hline
	action & get MAMA service basic informations (release date / version / ...) \\
	\hline
	output & MAMA informations (\texttt{JSON} / \texttt{XML} / \texttt{text} formats). \\
	\hline
	cause of failure & none. \\
	\hline
\end{tabular}
\newline

\subsubsection{Create a new user}
\label{postUser}
\begin{tabular}{ | l | l | }
	\hline
	HTTP verb and URI & \texttt{POST /user} \\
	\hline
	action & create a new user; sent a notification email. \\
	\hline
	mandatory POST parameters & ``email'' (used as login later, can not be changed) \\
	\space & ``password'' \\
	\hline
	optional POST parameters & ``firstName'' \\
	\space & ``lastName''  \\
	\space & ``phoneGroup'' (int)\footnote{\cf: \url{https://countrycode.org/}} \\
	\space & ``phoneNumber'' \\
	\space & ``laboratoryOrCompagny'' \\
	\space & ``workplaceAddress'' \\
	\space & ``typeOfLaboratoryOrCompany'' \\
	\space & ``lang'' (the language for the email sent, default value: ``en'') \\	
	\hline
	output & the ID of the new user or ``-1'' if error (\texttt{JSON} / \texttt{XML} / \texttt{text} formats). \\
	\hline
	cause of failure & an user with the same login already exist in the system. \\
	\hline
\end{tabular}
\newline

\subsubsection{Create a new Token}
\label{postToken}
Note: if the user already have a valid token, it will return it and extends his validity date. \\
\begin{tabular}{ | l | l | }
	\hline
	HTTP verb and URI & \texttt{POST /token} \\
	\hline
	action & create a new token; return this new token with his decrepit date. \\
	\hline
	mandatory POST parameters & ``login'' (your email or LDAP login) \\
	\space & ``password'' \\
	\hline
	output & the new token with his decrepit date \\
	\space & your account basic informations (\texttt{JSON} / \texttt{XML} / \texttt{text} formats). \\
	\hline
	cause of failure & wrong login / password combination. \\
	\hline
\end{tabular}
\newline

\subsubsection{Reset your password (only for none-LDAP users)}
\begin{tabular}{ | l | l | }
	\hline
	HTTP verb and URI & \texttt{PUT /reset-password} \\
	\hline
	action & generate a new password for your account en send it \via email. \\
	\hline
	mandatory PUT parameters & ``login'' (your email) \\
	\hline
	optional  PUT parameters & ``lang'' (the language for the email sent, default value: ``en'') \\
	\hline
	output & ``true'' if the reset password email has been sent; \\
	\space & ``false'' otherwise (boolean value, in \texttt{JSON} / \texttt{XML} / \texttt{text} formats). \\
	\hline
	cause of failure & wrong login (not in the system) / \\
	\space & the system can not write an email to this address. \\
	\hline
\end{tabular}
\newline

%GET /keywords
\subsubsection{Get the list of keywords}
\hspace*{\parindent}
These keywords are used when a user create or edit a project
(you can associate up to three keywords per project).
\newline
\begin{tabular}{ | l | l | }
	\hline
	HTTP verb and URI & \texttt{GET /keywords} \\
	\hline
	action & list available keywords. \\
	\hline
	output & list of keywords and their ids \\
	\space & (in \texttt{JSON} / \texttt{XML} / \texttt{text} formats). \\
	\hline
	cause of failure & none. \\
	\hline
\end{tabular}
\newline

%GET /subkeywords
\subsubsection{Get the list of sub-keywords}
\hspace*{\parindent}
These sub-keywords are used when a user create or edit a project
(you can associate up to five sub-keywords per project).
\newline
\begin{tabular}{ | l | l | }
	\hline
	HTTP verb and URI & \texttt{GET /subkeywords} \\
	\hline
	action & list available sub-keywords. \\
	\hline
	output & list of sub-keywords and their ids \\
	\space & (in \texttt{JSON} / \texttt{XML} / \texttt{text} formats). \\
	\hline
	cause of failure & none. \\
	\hline
\end{tabular}
\newline

%GET /mth-platforms
\subsubsection{Get the list of MetaboHUB's platforms}
\hspace*{\parindent}
These platform are used when a user create or edit a project.
\newline
\begin{tabular}{ | l | l | }
	\hline
	HTTP verb and URI & \texttt{GET /mth-platforms} \\
	\hline
	action & list available metaboHUB's platforms. \\
	\hline
	output & list of metaboHUB's platforms and their ids \\
	\space & (in \texttt{JSON} / \texttt{XML} / \texttt{text} formats). \\
	\hline
	cause of failure & none. \\
	\hline
\end{tabular}
\newline

\subsection{``User'' rights}

%GET /user/\{id\}
\subsubsection{Get a user informations}
\begin{tabular}{ | l | l | }
	\hline
	HTTP verb and URI & \texttt{GET /user/\{id\}} \\
	\hline
	action & get all informations about a user (identified \via user's \texttt{id}). \\
	\hline
	mandatory GET parameter & ``token'' (value: user's token) \\
	\hline
	output & list all user's properties / settings.  \\
	\space & (in \texttt{JSON} / \texttt{XML} / \texttt{text} formats). \\
	\hline
	cause of failure & the user is not allowed to see requested user informations. \\
	\space & the user token is deprecated. \\
	\hline
\end{tabular}
\newline

%PUT /user/\{id\}
\subsubsection{Edit user's informations}
\label{putUsersInformations}
\begin{tabular}{ | l | l | }
	\hline
	HTTP verb and URI & \texttt{PUT /user\{id\}} \\
	\hline
	action & edit a user informations. \\
	\hline
	mandatory GET parameter & ``token'' (value: user's token) \\
	\hline
	mandatory PUT parameters & ``firstName'' \\
	\space & ``lastName''  \\
	\space & ``phoneGroup'' (int)\footnote{\cf: \url{https://countrycode.org/}} \\
	\space & ``phoneNumber'' \\
	\space & ``laboratoryOrCompagny'' \\
	\space & ``workplaceAddress'' \\
	\space & ``typeOfLaboratoryOrCompany'' \\	
	\space & ``emailNotification'' (must be ``true'' or ``false'') \\
	\space & ``emailAlertNewUserAccount'' (must be ``true'' or ``false'' \footnote{only work if you are an ``admin''})) \\
	\space & ``emailAlertNewProject'' (must be ``true'' or ``false'' \footnote{only work if you are a ``project manager'' / ``admin''}) \\
	\space & ``emailAlertNewEventFollowedProject'' (must be ``true'' or ``false'' \footnote{only work if you are a ``project manager'' / ``admin''}) \\
	\space & ``emailAlertNewMessage'' (must be ``true'' or ``false'') \\	
	\hline
	optional PUT parameters & ``password'' and ``old-password'' only if you want to update your password. \\
	\space & ``lang'' (the language for the email sent, default value: ``en'') \\	
	\hline
	output & ``true'' if the user profile has been successfully updated; \\
	\space & ``false'' otherwise (boolean value, in \texttt{JSON} / \texttt{XML} / \texttt{text} formats). \\
	\hline
	cause of failure & the user token is deprecated. \\
	\hline
\end{tabular}
\newline

%GET /projects
\subsubsection{Get / list projects}
\begin{tabular}{ | l | l | }
	\hline
	HTTP verb and URI & \texttt{GET /projects} \\
	\hline
	action & list all user's projects. \\
	\hline
	mandatory GET parameter & ``token'' (value: user's token) \\
	\hline
	optional GET parameter & ``userFilter'' add a filter (value: ``owner'' or ``inCharge'' or ``involved'') \\
	\space & ``start'' list from this value (value: integer) \\
	\space & ``limit'' number of result max to show (value: integer) \\
	\space & ``from'' list from this date (format: \texttt{YYYY-MM-DD}) \\
	\space & ``to'' list until this date (format: \texttt{YYYY-MM-DD}) \\
	\space & ``to'' number of result max to show (value: integer) \\
%	\space & ``deleted'' add a filter (value: ``true'' or ``false'') \\
	\space & ``status'' filter only projects with this status \footnote{can be ``rejected'' or ``waiting'' or ``completed'' or ``accepted'' or ``assigned'' or ``running'' or ``blocked'' or ``archived''. } \\
	\space & ``keywords'' search projects with title like this value (value: string) \\
	\space & ``order'' sort projects by id ascending / descending (value: ``desc'' or ``asc'')  \\
	\hline
	output & list all user's projects.  \\
	\space & (in \texttt{JSON} / \texttt{XML} / \texttt{text} / \texttt{XLS} formats \footnote{the \texttt{XLS} format only works if it is send as GET parameter.}). \\
	\hline
	cause of failure & the user token is deprecated. \\
	\hline
\end{tabular}
\newline

%GET /projects-stats
\subsubsection{Count projects per status}
\begin{tabular}{ | l | l | }
	\hline
	HTTP verb and URI & \texttt{GET /projects-stats} \\
	\hline
	action & count all user's projects. \\
	\hline
	mandatory GET parameter & ``token'' (value: user's token) \\
	\hline
	optional GET parameter & ``userFilter'' add a filter (value: ``owner'' or ``inCharge'' or ``involved'') \\
%	\space & ``deleted'' add a filter (value: ``true'' or ``false'') \\
	\space & ``status'' filter only projects with this status \footnote{can be ``rejected'' or ``waiting'' or ``completed'' or ``accepted'' or ``assigned'' or ``running'' or ``blocked'' or ``archived''. } \\
	\space & ``keywords'' search projects with title like this value (value: string) \\
	\space & ``from'' count projects from this date (format: \texttt{YYYY-MM-DD}) \\
	\space & ``to'' count projects until this date (format: \texttt{YYYY-MM-DD}) \\
	\hline
	output & list all user's projects.  \\
	\space & (in \texttt{JSON} / \texttt{XML} / \texttt{text} formats). \\
	\hline
	cause of failure & the user token is deprecated. \\
	\hline
\end{tabular}
\newline

%GET /events
\subsubsection{Get / list events}
\begin{tabular}{ | l | l | }
	\hline
	HTTP verb and URI & \texttt{GET /events} \\
	\hline
	action & list all user's projects. \\
	\hline
	mandatory GET parameter & ``token'' (value: user's token) \\
	\hline
	optional GET parameter & ``filter'' add a filter (value: ``users'' or ``projects'' or ``admin'') \footnote{filter type of events; ``admin only works for administrators.''} \\
	\space & ``start'' list from this value (value: integer) \\
	\space & ``limit'' number of result max to show (value: integer) \\
	\space & ``from'' list events from this date (format: \texttt{YYYY-MM-DD}) \\
	\space & ``to'' list events until this date (format: \texttt{YYYY-MM-DD}) \\
%	\space & ``deleted'' add a filter (value: ``true'' or ``false'') \\
	\space & ``order'' sort events by id ascending / descending (value: ``desc'' or ``asc'')  \\
	\hline
	output & list all user's events (projects + his profile).  \\
	\space & (in \texttt{JSON} / \texttt{XML} / \texttt{text} formats). \\
	\hline
	cause of failure & the user token is deprecated. \\
	\hline
\end{tabular}
\newline

%GET /messages
\subsubsection{Get / list messages}
\begin{tabular}{ | l | l | }
	\hline
	HTTP verb and URI & \texttt{GET /messages} \\
	\hline
	action & list all user's messages. \\
	\hline
	mandatory GET parameter & ``token'' (value: user's token) \\
	\hline
	optional GET parameter & ``projectID'' return only messages linked to this project \\
	\space & ``userFilter'' return only messages send or received \\ 
	\space & (value: ``sender'' or ``receiver'' (default)) \\
	\space & ``projectFilter'' return only messages on project \\
	\space & (value: ``owner'' (only value possible for users) or ``inCharge'' / \\
	\space & ``involved'' (only project-manager / admin)) \\
	\space & ``noUsers'' return only messages on projects \\ 
	\space & (value: ``true'' or ``false'' (default)) \\
	\space & ``start'' list from this value (value: integer) \\
	\space & ``limit'' number of result max to show (value: integer) \\
	\space & ``from'' list messages from this date (format: \texttt{YYYY-MM-DD}) \\
	\space & ``to'' list messages until this date (format: \texttt{YYYY-MM-DD}) \\
%	\space & ``deleted'' add a filter (value: ``true'' or ``false'') \\
	\space & ``order'' sort events by id ascending / descending (value: ``desc'' or ``asc'')  \\
	\hline
	output & list all user's messages.  \\
	\space & (in \texttt{JSON} / \texttt{XML} / \texttt{text} formats). \\
	\hline
	cause of failure & the user token is deprecated. \\
	\hline
\end{tabular}
\newline

%GET /appointments
\subsubsection{Get / list appointments}
\begin{tabular}{ | l | l | }
	\hline
	HTTP verb and URI & \texttt{GET /appointments} \\
	\hline
	action & list all user's appointments. \\
	\hline
	mandatory GET parameter & ``token'' (value: user's token) \\
	\hline
	optional GET parameter & ``projectID'' return only appointments linked to this project \\
	\space & ``userFilter'' return only appointments from or to you \\ 
	\space & (value: ``from'' or ``to'' (default: all)) \\
	\space & ``start'' list from this value (value: integer) \\
	\space & ``limit'' number of result max to show (value: integer) \\
	\space & ``filter'' fetch only past or ongoing appointments \\
	\space & (value: ``ongoing'' or ``past'' (default: all)) \\
%	\space & ``deleted'' add a filter (value: ``true'' or ``false'') \\
	\space & ``order'' sort events by id ascending / descending (value: ``desc'' or ``asc'')  \\
	\hline
	output & list all user's appointments.  \\
	\space & (in \texttt{JSON} / \texttt{XML} / \texttt{text} formats). \\
	\hline
	cause of failure & the user token is deprecated. \\
	\hline
\end{tabular}
\newline

%POST /project
\subsubsection{Create a new project}
\begin{tabular}{ | l | l | }
	\hline
	HTTP verb and URI & \texttt{POST /project} \\
	\hline
	action & create a new project. \\
	\hline
	mandatory GET parameter & ``token'' (value: user's token) \\
	\hline
	mandatory POST parameters & ``title'' (can not be edited later) \\
	\space & ``interestInMthCollaboration'' free text \\		
	\space & ``demandTypeEqProvisioning'' boolean value \\
	\space & ``demandTypeCatalogAllowance'' boolean value \\
	\space & ``demandTypeFeasibilityStudy'' boolean value \\
	\space & ``demandTypeFormation'' boolean value \\
%%	// $samplesNumber,
	\space & ``samplesNumber'' number of sample \footnote{only relevant if ``demandTypeCatalogAllowance'' is ``true''; can be ``50 or fewer'' or ``51 to 100'' or ``101 to 500'' or ``more than 501'' } \\
	\space & ``cloudWords'' list of integers, comma separated \footnote{each integer matching a keywords’s ID; see \texttt{GET /keywords} method}. \\
	\space & ``mthPlatforms'' list of integers, comma separated \footnote{each integer matching an MTH platform’s ID; see \texttt{GET /mth-platforms} method}. \\
	\space & ``copartner'' boolean value \\
	\space & ``targeted'' boolean value \\
	\hline
	optional POST parameters & ``scientificContext'' free text \\	
	\space & ``scientificContextFile'' string return by \texttt{POST /project-file} method. \\
	\space & ``project\_financed'' boolean value  \\
	\space & ``project\_inProvisioning'' boolean value \\
	\space & ``project\_onOwnSupply'' boolean value \\
	\space & ``project\_notFinanced'' boolean value \\
%%	// financialContextBis: 
	\space & ``project\_financedEU'' boolean value \footnote{only relevant if ``project\_financed'' or ``project\_inProvisioning'' is ``true'')} \\
	\space & ``project\_financedANR'' boolean value \footnote{only relevant if ``project\_financed'' or ``project\_inProvisioning'' is ``true'')} \\
	\space & ``project\_financedNational'' boolean value \footnote{only relevant if ``project\_financed'' or ``project\_inProvisioning'' is ``true'')} \\
	\space & ``project\_financedRegional'' boolean value \footnote{only relevant if ``project\_financed'' or ``project\_inProvisioning'' is ``true'')} \\
	\space & ``project\_financedCompanyTutorship'' boolean value \footnote{only relevant if ``project\_financed'' or ``project\_inProvisioning'' is ``true'')} \\
	\space & ``project\_financedOther'' boolean value \footnote{only relevant if ``project\_financed'' or ``project\_inProvisioning'' is ``true'')} \\
	\space & ``financialContextOther'' value: string \footnote{ignored if ``project\_financedOther'' is ``false''} \\
	\hline
	output & the ID of the new project, \\
	\space & or ``-1'' if failure (\texttt{JSON} / \texttt{XML} / \texttt{text} formats). \\
	\hline
	cause of failure & missing mandatory information(s) \\
	\space & the user token is deprecated. \\
	\space & the user is blocked. \\
	\hline
\end{tabular}

%POST /message
\subsubsection{Create a new message}
\begin{tabular}{ | l | l | }
	\hline
	HTTP verb and URI & \texttt{POST /message} \\
	\hline
	action & create a new message. \\
	\hline
	mandatory GET parameter & ``token'' (value: user's token) \\
	\hline
	mandatory POST parameters & ``message'' (can not be edited later) \\
	\space & ``projectID'' id of a project \\		
	\space & OR ``userID'' id of a user \\
	\hline
	output & the ID of the new message, \\
	\space & or ``-1'' if failure (\texttt{JSON} / \texttt{XML} / \texttt{text} formats). \\
	\hline
	cause of failure & missing mandatory information(s); \\
	\space & the user is not allowed to see / edit the project. \\
	\space & the user token is deprecated. \\
	\space & the user is blocked. \\
	\hline
\end{tabular}

%POST /message
\subsubsection{Create a new appointment}
\begin{tabular}{ | l | l | }
	\hline
	HTTP verb and URI & \texttt{POST /appointment} \\
	\hline
	action & create a new appointment. \\
	\hline
	mandatory GET parameter & ``token'' (value: user's token) \\
	\hline
	mandatory POST parameters & ``message'' (can not be edited later) \\
	\space & ``projectID'' id of a project \\		
	\space & ``toUser'' id of a user to fix appointment \\
	\space & ``dates'' appointment dates suggested \\
	\space & (format: ``yyyy-mm-dd HH:MM'', comma separated) \\
	\hline
	output & the ID of the new appointment, \\
	\space & or ``-1'' if failure (\texttt{JSON} / \texttt{XML} / \texttt{text} formats). \\
	\hline
	cause of failure & missing mandatory information(s);  \\
	\space & the user is not allowed to see / edit the project. \\
	\space & the user token is deprecated. \\
	\space & the user is blocked. \\
	\hline
\end{tabular}

%GET /project/\{id\}
\subsubsection{Get a project informations}
\begin{tabular}{ | l | l | }
	\hline
	HTTP verb and URI & \texttt{GET /project/\{id\}} \\
	\hline
	action & get all informations about a project (identified \via project’s \texttt{id}). \\
	\hline
	mandatory GET parameter & ``token'' (value: user's token) \\
	\hline
	output & list all project's properties.  \\
	\space & (in \texttt{JSON} / \texttt{XML} / \texttt{text} formats). \\
	\hline
	cause of failure & the user is not allowed to see requested informations. \\
	\space & the user token is deprecated. \\
	\hline
\end{tabular}
\newline

%GET /message/ID
\subsubsection{Get a message informations}
\begin{tabular}{ | l | l | }
	\hline
	HTTP verb and URI & \texttt{GET /message/\{id\}} \\
	\hline
	action & get informations about a message (identified \via message’s \texttt{id}). \\
	\hline
	mandatory GET parameter & ``token'' (value: user's token) \\
	\hline
	output & list all message's properties.  \\
	\space & (in \texttt{JSON} / \texttt{XML} / \texttt{text} formats). \\
	\hline
	cause of failure & the user is not allowed to see requested informations. \\
	\space & the user token is deprecated. \\
	\hline
\end{tabular}
\newline

%GET /appointment/ID
\subsubsection{Get an appointment informations}
\begin{tabular}{ | l | l | }
	\hline
	HTTP verb and URI & \texttt{GET /appointment/\{id\}} \\
	\hline
	action & get informations about a appointment (identified \via appointment’s \texttt{id}). \\
	\hline
	mandatory GET parameter & ``token'' (value: user's token) \\
	\hline
	output & list all appointment's properties.  \\
	\space & (in \texttt{JSON} / \texttt{XML} / \texttt{text} formats). \\
	\hline
	cause of failure & the user is not allowed to see requested informations. \\
	\space & the user token is deprecated. \\
	\hline
\end{tabular}
\newline

%PUT /project/\{id\}
\subsubsection{Edit project's informations}
\label{putprojects}
\begin{tabular}{ | l | l | }
	\hline
	HTTP verb and URI & \texttt{PUT /project\{id\}} \\
	\hline
	action & edit a project (identified \via project's id). \\
	\hline
	mandatory GET parameter & ``token'' (value: user's token) \\
	\hline
	optional PUT parameters & ``interestInMthCollaboration'' free text \\		
	\space & ``demandTypeEqProvisioning'' boolean value \\
	\space & ``demandTypeCatalogAllowance'' boolean value \\
	\space & ``demandTypeFeasibilityStudy'' boolean value \\
	\space & ``demandTypeFormation'' boolean value \\
%%	// $samplesNumber,
	\space & ``samplesNumber'' number of sample \footnote{only relevant if ``demandTypeCatalogAllowance'' is ``true''; can be ``50 or fewer'' or ``51 to 100'' or ``101 to 500'' or ``more than 501'' } \\
	\space & ``cloudWords'' list of integers, comma separated \footnote{each integer matching a keywords’s ID; see \texttt{GET /keywords} method}. \\
	\space & ``mthPlatforms'' list of integers, comma separated \footnote{each integer matching an MTH platform’s ID; see \texttt{GET /mth-platforms} method}. \\
	\space & ``copartner'' boolean value \\
	\space & ``targeted'' boolean value \\
	\space & ``scientificContext'' free text \\	
	\space & ``scientificContextFile'' string return by \texttt{POST /project-file} method. \\
	\space & ``project\_financed'' boolean value  \\
	\space & ``project\_inProvisioning'' boolean value \\
	\space & ``project\_onOwnSupply'' boolean value \\
	\space & ``project\_notFinanced'' boolean value \\
%%	// financialContextBis: 
	\space & ``project\_financedEU'' boolean value \footnote{only relevant if ``project\_financed'' or ``project\_inProvisioning'' is ``true'')} \\
	\space & ``project\_financedANR'' boolean value \footnote{only relevant if ``project\_financed'' or ``project\_inProvisioning'' is ``true'')} \\
	\space & ``project\_financedNational'' boolean value \footnote{only relevant if ``project\_financed'' or ``project\_inProvisioning'' is ``true'')} \\
	\space & ``project\_financedRegional'' boolean value \footnote{only relevant if ``project\_financed'' or ``project\_inProvisioning'' is ``true'')} \\
	\space & ``project\_financedCompanyTutorship'' boolean value \footnote{only relevant if ``project\_financed'' or ``project\_inProvisioning'' is ``true'')} \\
	\space & ``project\_financedOther'' boolean value \footnote{only relevant if ``project\_financed'' or ``project\_inProvisioning'' is ``true'')} \\
	\space & ``financialContextOther'' value: string \footnote{ignored if ``project\_financedOther'' is ``false''} \\
	\hline
	output & ``true'' if the project has been successfully updated; \\
	\space & ``false'' otherwise (boolean value, in \texttt{JSON} / \texttt{XML} / \texttt{text} formats). \\
	\hline
	cause of failure &the user is not allowed to edit thsi project \\
	\space & the user token is deprecated. \\
	\hline
\end{tabular}
\newline

%POST /project-file
\subsubsection{Upload a ``scientific context'' file}
\begin{tabular}{ | l | l | }
	\hline
	HTTP verb and URI & \texttt{POST /project-file} \\
	\hline
	action & upload a project's ``scientific context file''. \\
	\hline
	mandatory GET parameter & ``token'' (value: user's token) \\
	\hline
	mandatory POST parameter & ``file'' (the file to upload) \\
	\hline
	output & the success (``true'' or ``false'')  \\
	\space & and the attached file's remote identifier \footnote{used in ``scientificContextFile'' field during POST or PUT /project}.  \\
	\space & (in \texttt{JSON} / \texttt{XML} / \texttt{text} formats). \\
	\hline
	cause of failure & the user is not allowed to see requested informations. \\
	\space & the user token is deprecated. \\
	\hline
\end{tabular}
\newline

%GET /project-file/\{id\}
\subsubsection{Download a ``scientific context'' file}
\begin{tabular}{ | l | l | }
	\hline
	HTTP verb and URI & \texttt{GET /project-file/\{id\}} \\
	\hline
	action & download a project's ``scientific context file'' (identified \via project's \texttt{id}). \\
	\hline
	mandatory GET parameter & ``token'' (value: user's token) \\
	\hline
	output & success: download the attached file; \\
	\space & error: get cause of failure (in \texttt{JSON} / \texttt{XML} / \texttt{text} formats). \\
	\hline
	cause of failure & the user is not allowed to see requested informations. \\
	\space & the user token is deprecated. \\
	\hline
\end{tabular}
\newline

%DELETE /project-file/\{id\}
\subsubsection{Delete a ``scientific context'' file}
\begin{tabular}{ | l | l | }
	\hline
	HTTP verb and URI & \texttt{DELETE /project-file/\{id\}} \\
	\hline
	action & delete a project's ``scientific context file'' (identified \via project's \texttt{id}). \\
	\hline
	mandatory DELETE parameter & ``token'' (value: user's token) \\
	\hline
	output & success: delete the attached file; \\
	\space & error: get cause of failure (in \texttt{JSON} / \texttt{XML} / \texttt{text} formats). \\
	\hline
	cause of failure & the user is not allowed to delete these informations. \\
	\space & the user token is deprecated. \\
	\hline
\end{tabular}
\newline

%DELETE /token[/\{token\}
\subsubsection{Delete a user's token}
\begin{tabular}{ | l | l | }
	\hline
	HTTP verb and URI & \texttt{DELETE /token/\{token\_value\}} \\
	\hline
	action & delete a token. \\
	\hline
	output & ``true'' if the token has been successfully deleted; \\
	\space & ``false'' otherwise (boolean value, in \texttt{JSON} / \texttt{XML} / \texttt{text} formats). \\
	\hline
	cause of failure & token already deleted. \\
	\hline
\end{tabular}
\newline

\subsection{``Project manager'' rights}
\hspace*{\parindent}
A project manager gets all ``users'' basic rights, some with more options and special services.
\newline

\hspace*{\parindent}
GET projects / GET projects-stats: get ALL projects
\newline

\subsubsection{Put Projects special options}
\begin{tabular}{ | l | l | }
	\hline
	HTTP verb and URI & \texttt{PUT /project\{id\}} \\
	\hline
	action & edit a project (identified \via project's id). \\
	\hline
	mandatory GET parameter & ``token'' (value: user's token) \\
	\hline
	optional PUT parameters & all fileds in ``Edit project’s informations'' (see \ref{putprojects} page \pageref{putprojects}) \\		
	\space & ``status'' String value \footnote{can be ``rejected'' or ``waiting'' or ``completed'' or ``accepted'' or ``assigned'' or ``running'' or ``blocked'' or ``archived''. } \\
	\space & ``analystInCharge'' ID of the projects manager (main) \\
	\space & ``analystInCharge'' IDs of the projects managers, comma separated \\
	\hline
	extra data as PUT parameters & ``extra\_adminContext'' Free text \\
	\space & ``extra\_geoContext'' Free text \\	
	\space & ``extra\_projectMaturity'' Free text \\
	\space & ``extra\_deadline'' Free text \\
	\space & ``extra\_budget'' Free text \\
	\space & ``extra\_userNeeds'' Free text \\
	\space & ``extra\_laboType'' must be \texttt{private} or \texttt{public} or \texttt{privatepublic} \\
	\space & ``extra\_hdykm\_friend'' \footnote{hdykm: How Did You Knew MetaboHUB} \texttt{true} or \texttt{false} \\
	\space & ``extra\_hdykm\_publication'' \texttt{true} or \texttt{false} \\
	\space & ``extra\_hdykm\_website'' \texttt{true} or \texttt{false} \\
	\space & ``extra\_hdykm\_searchEngine'' \texttt{true} or \texttt{false} \\
	\hline
	output & ``true'' if the project has been successfully updated; \\
	\space & ``false'' otherwise (boolean value, in \texttt{JSON} / \texttt{XML} / \texttt{text} formats). \\
	\hline
	cause of failure &the user is not allowed to edit thsi project \\
	\space & the user token is deprecated. \\
	\hline
\end{tabular}
\newline

\subsubsection{Block or Pause a Project}
%PUT 
\begin{tabular}{ | l | l | }
	\hline
	HTTP verb and URI & \texttt{PUT /stop-project\{id\}} \\
	\hline
	action & block or pause a project (identified \via project's id). \\
	\hline
	mandatory GET parameter & ``token'' (value: user's token) \\
	\hline
	optional PUT parameters & ``blockedCase'' String value \footnote{can be ``waiting\_for\_samples'' or ``waiting\_for\_service\_user\_answer'' or ``waiting\_for\_provisioning''.} \\		
	\space & ``rejectedCase'' String value \footnote{can be ``too\_expensive'' or ``too\_long\_delays'' or ``outside\_our\_skill\_sphere'' or ``non\_prioritary\_rad'' or ``incompatible\_deadline'' or ``too\_many\_samples'' or ``transfered\_to\_privilegied\_mth\_partner'' or ``not\_funded''.} \\
	\space & ``stoppedReason'' Free text. \\
	\hline
	output & ``true'' if the project has been successfully updated; \\
	\space & ``false'' otherwise (boolean value, in \texttt{JSON} / \texttt{XML} / \texttt{text} formats). \\
	\hline
	cause of failure &the user is not allowed to edit thsi project \\
	\space & the user token is deprecated. \\
	\hline
\end{tabular}
\newline

% GET /projects-managers
\subsubsection{List all ``project managers'' and ``admin''}
\begin{tabular}{ | l | l | }
	\hline
	HTTP verb and URI & \texttt{GET /projects-managers} \\
	\hline
	action & get all Projects Managers \footnote{and admins, because they are projects managers too!}. \\
	\hline
	mandatory GET parameter & ``token'' (value: user's token) \\
	\hline
	output & the list of MAMA's Projects Managers and Admins \\ 
	\space & (in \texttt{JSON} / \texttt{XML} / \texttt{text} formats). \\
	\hline
	cause of failure & the user token is deprecated. \\
	\hline
\end{tabular}
\newline

\subsection{``Admin'' rights}

%GET /users
\subsubsection{Get / list users}
\begin{tabular}{ | l | l | }
	\hline
	HTTP verb and URI & \texttt{GET /users} \\
	\hline
	action & list users from the system. \\
	\hline
	mandatory GET parameter & ``token'' (value: user's token) \\
	\hline
	optional GET parameters & ``keyword'' return only users matching this value \footnote{either in ``firstname'', ``lastname'' or ``email''} \\
	\space & ``start'' list from this value (value: integer) \\
	\space & ``limit'' number of result max to show (value: integer) \\
	\space & ``order'' can be ``ASC'' or `DESC`'' \\
	\space &  ``status'' standardized String \footnote{must be ``blocked'' or ``user'' or ``project\_manager'' or ``admin''} \\		
	\hline
	output & ``true'' if the user rights has been successfully updated; \\
	\space & ``false'' otherwise (boolean value, in \texttt{JSON} / \texttt{XML} / \texttt{text} formats). \\
	\hline
	cause of failure & the user token is deprecated. \\
	\hline
\end{tabular}
\newline

%GET /users-stats
\subsubsection{Count users per status / rights}
\begin{tabular}{ | l | l | }
	\hline
	HTTP verb and URI & \texttt{GET /users-stats} \\
	\hline
	action & count users per status / rights in the system. \\
	\hline
	mandatory GET parameter & ``token'' (value: user's token) \\
	\hline
	optional GET parameters & ``keyword'' return only users matching this value \footnote{either in ``firstname'', ``lastname'' or ``email''} \\	
	\hline
	output & ``true'' if the user rights has been successfully updated; \\
	\space & ``false'' otherwise (boolean value, in \texttt{JSON} / \texttt{XML} / \texttt{text} formats). \\
	\hline
	cause of failure & the user token is deprecated. \\
	\hline
\end{tabular}
\newline

%PUT /user-right/\{id\}
\subsubsection{Edit user's rights}
\begin{tabular}{ | l | l | }
	\hline
	HTTP verb and URI & \texttt{PUT /user-right/\{id\}} \\
	\hline
	action & edit a user rights. \\
	\hline
	mandatory GET parameter & ``token'' (value: user's token) \\
	\hline
	mandatory PUT parameter & ``statusRight'' standardized String \footnote{must be ``blocked'' or ``user'' or ``project\_manager'' or ``admin''} \\		
	\hline
	output & ``true'' if the user rights has been successfully updated; \\
	\space & ``false'' otherwise (boolean value, in \texttt{JSON} / \texttt{XML} / \texttt{text} formats). \\
	\hline
	cause of failure & the user token is deprecated. \\
	\hline
\end{tabular}
\newline

%GET /server-load
\subsubsection{Get sever load}
\begin{tabular}{ | l | l | }
	\hline
	HTTP verb and URI & \texttt{GET /server-load} \\
	\hline
	action & get informations about server (load, memory, ...). \\
	\hline
	mandatory GET parameter & ``token'' (value: user's token) \\
	\hline
	output & list server load properties.  \\
	\space & (in \texttt{JSON} / \texttt{XML} / \texttt{text} formats). \\
	\hline
	cause of failure & the user is not allowed to see requested informations. \\
	\space & the user token is deprecated. \\
	\hline
\end{tabular}
\newline

%POST /keyword
\subsubsection{Create a new keyword}
\begin{tabular}{ | l | l | }
	\hline
	HTTP verb and URI & \texttt{POST /keyword} \\
	\hline
	action & create a new keyword. \\
	\hline
	mandatory GET parameter & ``token'' (value: user's token) \\
	\hline
	mandatory POST parameter & ``keyword'' (value: name of the new keyword) \\
	\hline
	output & the success (``true'' or ``false'')  \\
	\space & the id of the new keyword.  \\
	\space & (in \texttt{JSON} / \texttt{XML} / \texttt{text} formats). \\
	\hline
	cause of failure & the user is not allowed to see requested informations. \\
	\space & the user token is deprecated / this keyword already exists. \\
	\hline
\end{tabular}
\newline

\subsubsection{Create a new sub-keyword}
\begin{tabular}{ | l | l | }
	\hline
	HTTP verb and URI & \texttt{POST /subkeyword} \\
	\hline
	action & create a new sub-keyword. \\
	\hline
	mandatory GET parameter & ``token'' (value: user's token) \\
	\hline
	mandatory POST parameter & ``keyword'' (value: name of the new sub-keyword) \\
	\hline
	output & the success (``true'' or ``false'')  \\
	\space & the id of the new sub-keyword.  \\
	\space & (in \texttt{JSON} / \texttt{XML} / \texttt{text} formats). \\
	\hline
	cause of failure & the user is not allowed to see requested informations. \\
	\space & the user token is deprecated / this sub-keyword already exists. \\
	\hline
\end{tabular}
\newline

%PUT /keyword/1
\subsubsection{Edit a keyword}
\begin{tabular}{ | l | l | }
	\hline
	HTTP verb and URI & \texttt{PUT /keyword/\{id\}} \\
	\hline
	action & update a keyword (identified \via its \texttt{id}). \\
	\hline
	mandatory GET parameter & ``token'' (value: user's token) \\
	\hline
	optional PUT parameters & ``keyword'' (value: new name for the keyword) \\
	\space & ``deleted'' (value: TRUE or FALSE) \\
	\hline
	output & the success (``true'' or ``false'')  \\
	\space & (in \texttt{JSON} / \texttt{XML} / \texttt{text} formats). \\
	\hline
	cause of failure & the user is not allowed to see requested informations. \\
	\space & the user token is deprecated / this keyword already exists. \\
	\hline
\end{tabular}
\newline

\subsubsection{Edit a sub-keyword}
\begin{tabular}{ | l | l | }
	\hline
	HTTP verb and URI & \texttt{PUT /subkeyword/\{id\}} \\
	\hline
	action & update a sub-keyword (identified \via its \texttt{id}). \\
	\hline
	mandatory GET parameter & ``token'' (value: user's token) \\
	\hline
	optional PUT parameters & ``keyword'' (value: new name for the sub-keyword) \\
	\space & ``deleted'' (value: TRUE or FALSE) \\
	\hline
	output & the success (``true'' or ``false'')  \\
	\space & (in \texttt{JSON} / \texttt{XML} / \texttt{text} formats). \\
	\hline
	cause of failure & the user is not allowed to see requested informations. \\
	\space & the user token is deprecated / this sub-keyword already exists. \\
	\hline
\end{tabular}
\newline

%POST /mth-platform
\subsubsection{Create a new MetaboHUB's platform}
\begin{tabular}{ | l | l | }
	\hline
	HTTP verb and URI & \texttt{POST /mth-platform} \\
	\hline
	action & create a new MetaboHUB's platform. \\
	\hline
	mandatory GET parameter & ``token'' (value: user's token) \\
	\hline
	mandatory POST parameter & ``mthPlatform'' (value: name of the new MetaboHUB's platform) \\
	\hline
	output & the success (``true'' or ``false'')  \\
	\space & the id of the new MetaboHUB's platform.  \\
	\space & (in \texttt{JSON} / \texttt{XML} / \texttt{text} formats). \\
	\hline
	cause of failure & the user is not allowed to see requested informations. \\
	\space & the user token is deprecated \\
	\space & this MetaboHUB's platform name already exists. \\
	\hline
\end{tabular}
\newline

%PUT /mth-platform
\subsubsection{Edit a MetaboHUB's platform}
\begin{tabular}{ | l | l | }
	\hline
	HTTP verb and URI & \texttt{PUT /mth-platform\{id\}} \\
	\hline
	action & edite a MetaboHUB's platform (identified \via its \texttt{id}). \\
	\hline
	mandatory GET parameter & ``token'' (value: user's token) \\
	\hline
	optional POST parameter & ``mthPlatform'' (value: new name of the new MetaboHUB's platform) \\
	\hline
	output & the success (``true'' or ``false'')  \\
	\space & (in \texttt{JSON} / \texttt{XML} / \texttt{text} formats). \\
	\hline
	cause of failure & the user is not allowed to see requested informations. \\
	\space & the user token is deprecated \\
	\space & this MetaboHUB's platform name already exists. \\
	\hline
\end{tabular}
\newline

\subsubsection{Get XLS file with all statistics}
\hspace*{\parindent}
This method require ``\texttt{xls}'' for ``\texttt{format}'' Get parameter or ``\texttt{application/xls}'' as value for ``\texttt{Accept}'' HTTP Header.
\\~\\
\begin{tabular}{ | l | l | }
	\hline
	HTTP verb and URI & \texttt{GET /server-load} \\
	\hline
	action & get many stastistics informations about MAMA's projects and users. \\
	\hline
	mandatory GET parameter & ``token'' (value: user's token) \\
	\hline
	output & The XLS file itself! \\
	\hline
	cause of failure & the user is not allowed to see requested informations. \\
	\space & the user token is deprecated. \\
	\hline
\end{tabular}
\newline

\subsubsection{Get a specific Project statistic}
\hspace*{\parindent}
Count all projects matching the filters
\\~\\
\begin{tabular}{ | l | l | }
	\hline
	HTTP verb and URI & \texttt{GET /projects-statistics} \\
	\hline
	action & get a specific project statistic. \\
	\hline
	mandatory GET parameter & ``token'' (value: user's token) \\
	\hline
	optional GET parameters & ``from'' (filter only projects with a creation date after this value) \\
	\space & ``to'' (filter only projects with a creation date before this value) \\
	\space & ``isStatus'' filter only projects with this status \\
	\space & ~~(can take one of the standardized values \footnote{``rejected'' or ``waiting'' or ``completed'' or ``accepted'' or ``assigned'' or ``running'' or ``blocked'' or ``archived''}) \\
	\space & ``isNotStatus'' filter only projects with not this status \\
	\space & ``isPlatForm'' filter only projects link to this MTH's platform \footnote{identified by its ID} \\
	\space & ``isNotPlatForm'' filter only projects NOT link to this MTH's platform  \\
	\space & ``isKeyword'' filter only projects link to this keyword \footnote{identified by its ID} \\
	\space & ``isNotKeyword'' filter only projects NOT link to this keyword \\
	\space & ``isType'' filter only project with this ``type'' value \\
	\space & ~~(can take the values ``eq\_prov'', ``cat\_allo'', ... \\
	\space & ~~separtate values by commas) \\
	\space & ``isNotType'' filter only project with NOT this ``type'' value \\
	\space & ``isFinanced'' filter only project with this ``financed'' value \\
	\space & ~~(can take the values ``financed eq\_prov'', ``provisioning'', ...) \\
	\space & ``isNotFinanced'' filter only project with NOT this ``financed'' value \\
	\space & ``isOwner'' filter only project with this ``owner'' ID(s) \\	
	\space & ~~(values separtate values by commas) \\	
	\space & ``isNotOwner'' filter only project with NOT this ``owner'' ID(s) \\	
	\space & ~~(values separtate values by commas) \\	
	\space & ``group'' choose a value to group your count; possible values:  \\
	\space & ~~ ``status'', ``sample\_number'', ``mthPF'', ``keywords'', ``subkeywords'', \\
	\space & ~~ ``type'', ``targeted'', ``copartner'', ``financial'' \\
	\hline
	output & the value(s) of the requested statistic \\
	\space & as \texttt{JSON} / \texttt{XML} / \texttt{text} formats \\
	\hline
	cause of failure & the user is not allowed to see requested informations. \\
	\space & the user token is deprecated. \\
	\hline
\end{tabular}
\newline

\subsubsection{Get a specific Project Extra statistic}
\hspace*{\parindent}
Count all projects matching the filters
\\~\\
\begin{tabular}{ | l | l | }
	\hline
	HTTP verb and URI & \texttt{GET /extra-data-statistics} \\
	\hline
	action & get a specific project statistic. \\
	\hline
	mandatory GET parameter & ``token'' (value: user's token) \\
	\hline
	optional GET parameters & ``from'' (filter only projects with a creation date after this value) \\
	\space & ``to'' (filter only projects with a creation date before this value) \\
	\space & ``isProjectLaboType'' filter only projects with this lab. value \\
	\space & ~~(can take one of the following values: ``public'', ``private'', ``public\_private'') \\
	\space & ``isNotProjectLaboType'' filter only projects with NOT this lab. value \\
	\space & ``group'' choose a value to group your count \\
	\space & ~~(possible value: ``laboratory'') \\
	\hline
	output & the value(s) of the requested statistic \\
	\space & as \texttt{JSON} / \texttt{XML} / \texttt{text} formats \\
	\hline
	cause of failure & the user is not allowed to see requested informations. \\
	\space & the user token is deprecated. \\
	\hline
\end{tabular}
\newline

\subsubsection{Get a specific User statistic}
\hspace*{\parindent}
Count all users matching the filters
\\~\\
\begin{tabular}{ | l | l | }
	\hline
	HTTP verb and URI & \texttt{GET /users-statistics} \\
	\hline
	action & get a specific user statistic. \\
	\hline
	mandatory GET parameter & ``token'' (value: user's token) \\
	\hline
	optional GET parameters & ``from'' (filter only users with a creation date after this value) \\
	\space & ``to'' (filter only users with a creation date before this value) \\
	\space & ``isUserLaboType'' filter only users with this lab. value \\
	\space & ~~(can take one of the following values: ``public'', ``private'', ``public\_private'') \\
	\space & ``isNotUserLaboType'' filter only users with NOT this lab. value \\
	\space & ``group'' choose a value to group your count \\
	\space & ~~(possible value: ``laboratory'') \\
	\hline
	output & the value(s) of the requested statistic \\
	\space & as \texttt{JSON} / \texttt{XML} / \texttt{text} formats \\
	\hline
	cause of failure & the user is not allowed to see requested informations. \\
	\space & the user token is deprecated. \\
	\hline
\end{tabular}
\newline

\subsubsection{Clean unused uploaded files}
\begin{tabular}{ | l | l | }
	\hline
	HTTP verb and URI & \texttt{PUT /admin/clean-uploaded-files} \\
	\hline
	action & clean unused uploaded files. \\
	\hline
	mandatory GET parameter & ``token'' (value: user's token) \\
	\hline
	output & the success (``true'' or ``false'') \\
	\space & the number of deleted files \\
	\space & as \texttt{JSON} / \texttt{XML} / \texttt{text} formats \\
	\hline
	cause of failure & the user is not allowed run the method. \\
	\space & the user token is deprecated. \\
	\hline
\end{tabular}
\newline

\subsubsection{Flag users as inactive}
\begin{tabular}{ | l | l | }
	\hline
	HTTP verb and URI & \texttt{PUT /admin/inactive-users} \\
	\hline
	action & flag user without recent activity as ``inactive''. \\
	\hline
	mandatory GET parameter & ``token'' (value: user's token) \\
	\hline
	mandatory PUT parameter & ``nbWeeks'' (value: the number of weeks threshold) \\
	\hline
	output & the success (``true'' or ``false'') \\
	\space & the number of inactivated users \\
	\space & as \texttt{JSON} / \texttt{XML} / \texttt{text} formats \\
	\hline
	cause of failure & the user is not allowed run the method. \\
	\space & the user token is deprecated. \\
	\hline
\end{tabular}
\newline

\subsubsection{Clean old tokens}
\begin{tabular}{ | l | l | }
	\hline
	HTTP verb and URI & \texttt{PUT /admin/clean-tokens} \\
	\hline
	action & clean old tokens. \\
	\hline
	mandatory GET parameter & ``token'' (value: user's token) \\
	\hline
	mandatory PUT parameter & ``nbHours'' (value: the number of hours threshold) \\
	\hline
	output & the success (``true'' or ``false'', \texttt{JSON} / \texttt{XML} / \texttt{text} formats) \\
	\hline
	cause of failure & the user is not allowed run the method. \\
	\space & the user token is deprecated. \\
	\hline
\end{tabular}
\newline

\subsubsection{Archive old projects}
\begin{tabular}{ | l | l | }
	\hline
	HTTP verb and URI & \texttt{PUT /admin/archive-projects} \\
	\hline
	action & archive old projects. \\
	\hline
	mandatory GET parameter & ``token'' (value: user's token) \\
	\hline
	mandatory PUT parameter & ``nbYears'' (value: the number of years threshold) \\
	\hline
	output & the success (``true'' or ``false'', \texttt{JSON} / \texttt{XML} / \texttt{text} formats) \\
	\hline
	cause of failure & the user is not allowed run the method. \\
	\space & the user token is deprecated. \\
	\hline
\end{tabular}
\newline

\subsubsection{Show log file}
\begin{tabular}{ | l | l | }
	\hline
	HTTP verb and URI & \texttt{GET /admin/show-logs/\{file\}} \\
	\hline
	action & get a LOG file. \\
	\hline
	mandatory GET parameter & ``token'' (value: user's token) \\
	\hline
	mandatory URI parameter & ``file'' (value: the log file name; either \\
	\space & ~~ ``weekly-mailler'', ``daily-mailler'' or ``monthly-users\_inactiver'') \\
	\hline
	optional GET parameter & ``from'' (value: the line to start) \\
	\space & ``to'' (value: the line to end) \\
	\hline
	output & the success (``true'' or ``false'', \texttt{JSON} / \texttt{XML} / \texttt{text} formats) \\
	\hline
	cause of failure & the user is not allowed run the method. \\
	\space & the user token is deprecated. \\
	\hline
\end{tabular}
\newline

\section{Web Service Client Examples}

\subsection{Java}
\hspace*{\parindent}
In the following example we use \href{https://code.google.com/p/json-simple/}{json-simple} library to parse JSON object. 
Feel free to use your favorite one!

\begin{lstlisting}[language=customJava,caption={Java request}]
package fr.metabohub.demo;

import java.io.BufferedReader;
import java.io.IOException;
import java.io.InputStream;
import java.io.InputStreamReader;
import java.io.Reader;
import java.net.URL;
import java.nio.charset.Charset;

import org.json.JSONException;
import org.json.JSONObject;

public class WebServiceClient {

	public static void main(String[] args) throws IOException, JSONException {
		JSONObject json = doGetJson("https://mama-rest.metabohub.fr/?format=json");
		System.out.println(json.toString());
	}

	public static JSONObject doGetJson(String url) throws IOException, JSONException {
		InputStream is = new URL(url).openStream();
		try {
			BufferedReader rd = new BufferedReader(new InputStreamReader(is, Charset.forName("UTF-8")));
			return new JSONObject(readBuffer(rd));
		} finally {
			is.close();
		}
	}

	private static String readBuffer(Reader rd) throws IOException {
		StringBuilder sb = new StringBuilder();
		int i;
		while ((i = rd.read()) != -1)
			sb.append((char) i);
		return sb.toString();
	}
	
}
\end{lstlisting}

\subsection{PHP}
\hspace*{\parindent}
Do not forget to check if the library \texttt{allow\_url\_fopen} is enabled in your configuration and if ``php-curl'' is installed on your system! 
\begin{lstlisting}[language=customPHP,caption={PHP request}]
<?php

// init
set_time_limit(0);
header("Content-Type: application/json; charset=UTF-8");

// call WebService
$json = file_get_contents('https://mama-rest.metabohub.fr/?format=json');

// do whatever you want
$obj = json_decode($json);
var_dump( $obj->mama );

?>
\end{lstlisting}

\subsection{Perl}
\hspace*{\parindent}
In this demo we do not parse the JSON result; 
Feel free to use your favourite library!
\begin{lstlisting}[language=customPerl,caption={Perl request}]
#!/usr/bin/perl
use strict;
use warnings;
use LWP::UserAgent;

# init
my $ws_url = "https://mama-rest.metabohub.fr";
my $format = "json";
my $rest_client = LWP::UserAgent->new;

# run
my $ws_query = $ws_url;
my $result = $rest_client->get($ws_query, ["format" => $format]);

# fail: print error code
if ( not $result->is_success) {
  print $result->status_line, "\n";
}

# success: print raw json
print $result->mama;

# end
exit 0;
\end{lstlisting}

\subsection{R-base}
\hspace*{\parindent}
This demo use \href{http://cran.r-project.org/web/packages/RCurl/index.html}{RCurl} library. 
We do not parse the JSON result so feel free to use your favourite library!
\begin{lstlisting}[language=customR,caption={R-base request}]
#!/usr/local/public/bin/Rscript --vanilla --slave --no-site-file

# load lib
library(RCurl)

# call WS
jsonData <- getForm("https://mama-rest.metabohub.fr", .params = c(format='json'))

# print
jsonData
\end{lstlisting}
\section{More} % mayo

% \subsection{MetaboHUB downloads}

\subsection{License}
\hspace*{\parindent}
This document and the code samples are under a \href{http://creativecommons.org/licenses/by-sa/4.0/}{Creative Commons License ``Attribution-ShareAlike 4.0 International'' (CC BY-SA 4.0)}. 
For more information please visit the \href{http://creativecommons.org/}{creative commons} website.
Sharing is caring! 

%\subsection{Disclaimer of Warranty}
%\hspace*{\parindent}
%TODO

\subsection{Contact}
\hspace*{\parindent}
To report a bug or ask for a new / specific method please contact us \via \href{mailto:contact-mama@metabohub.fr?subject=\%5BWebservice-REST\%20Issue\%5D}{contact-mama@metabohub.fr}. 
% NEED tickets management system
For generic questions about the MAMA service you can send use an email \via \href{mailto:contact-mama@metabohub.fr?subject=\%5BWebservice-REST\%5D}{contact-mama@metabohub.fr}. 

%\newpage
\newpage

%%% bibliographie %%%

%\bibliographystyle{unsrt}
%\bibliography{files/biblio}

\end{document}
